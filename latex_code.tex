\documentclass{article}
\usepackage{graphicx}
\usepackage{caption}
\usepackage{subcaption}

\begin{document}

\title{\textbf{PROGRAMMING PRACTICES \\
MINI PROJECT}}
\author{AKSHAY KUMAR VERMA}
\date{\today}

\maketitle

\newpage % Start the Introduction on a new page
\section{Introduction}
This project serves as one stop destination for all your needs related for competitive examinations. The current scenario in competitive world shows that the competition is increased too much whether it is in the field of JEE,NEET,Clat or it is about any higher engineering examinations like GATE.The AIM of this interface is to cater the candidates with all possible and the best resources for the respective competitive examinations they are going to phase. 

%\newpage % Start the Methodology on a new page
\section{Methodology}
The project is made up using the cpp programming language.This report also includes the other things as well like the\textit{ \textbf{profiling} }and \textit{\textbf{debugging}} operations perfomed on the code and the respective results obtained. 
\subsection{Flow of code}
 The flow of the code goes in the way that:\\
 1. There is a prompt for the category to choose for.\\
 2. On the basis of chosen category \textbf{viewBookdetails} function is called. Which displays the different sub available int categories.\\
 3. Then prompt is there to enter integer digit to choose which subject need to choose.\\
 4. Then on the basis of integer input the \textbf{viewSubDetails} function is called.\\
 5. Then it list all the best book resouces to follow for that subject.\\
 6.Then further either you can exit the terminal by entering 0 or can check for any other subject as well.\\
 7.This is how the code goes...\\
 \section{Scalability}
 This interface is a basic one but this can be scale to large scale by adding many more features such as adding tests,youtube resources, teaching assistants,doubt resolver and many more facilities like that...
 \newpage % Start the Results on a new page
\section{Debugging}
The process of identifying and removing errors from code is debugging.\\
Here is how the debbuging process was applied on the code and the results i got:\\

1: It shows that compilation was not successfull due to d was not declared in scope and also not defined properly\\
\begin{figure}[h]
  
  \begin{subfigure}{1.5\textwidth}
    \includegraphics[width=\linewidth]{Screenshot 2023-10-28 180842.png}
    \caption{Image 1}
  \end{subfigure}
 
\end{figure}
\\
2: It shows on debugging the d was declared in the another scope and on entering value of d the result getting was not the expected one.\\
\begin{figure}[h]
  \centering
  \includegraphics[width=1.4\textwidth]{Screenshot 2023-10-28 181835.png}
  \caption{Image 2}
\end{figure}

\newpage % Start the Discussion on a new page
\section{Output }
The generated output on running the code:
\begin{figure}[h]
  \centering
  \includegraphics[width=1.6\textwidth]{Screenshot 2023-10-29 202247.png}
  \caption{Image 11}
\end{figure}
\begin{figure}[h]
  \centering
  \includegraphics[width=1.6\textwidth]{Screenshot 2023-10-29 202304.png}
  \caption{Image 12}
\end{figure}
\newpage % Start the Discussion on a new page
\section{Profiling}
On profiling using different \textbf{\textit{gprof }}commands got the following flat profile and call graph:\\
\\
\begin{figure}[h]
  \centering
  \includegraphics[width=1.6\textwidth]{Screenshot 2023-10-28 160815.png}
  \caption{Image 3}
\end{figure}
\\
\begin{figure}[h]
  \centering
  \includegraphics[width=1.6\textwidth]{Screenshot 2023-10-28 160834.png}
  \caption{Image 4}
\end{figure}
\\
\begin{figure}[h]
  \centering
  \includegraphics[width=1.6\textwidth]{Screenshot 2023-10-28 160846.png}
  \caption{Image 5}
\end{figure}
\\
\begin{figure}[h]
  \centering
  \includegraphics[width=1.6\textwidth]{Screenshot 2023-10-28 160900.png}
  \caption{Image 6}
\end{figure}
\\
\begin{figure}[h]
  \centering
  \includegraphics[width=1.6\textwidth]{Screenshot 2023-10-28 160916.png}
  \caption{Image 7}
\end{figure}
\\
\begin{figure}[h]
  \centering
  \includegraphics[width=1.6\textwidth]{Screenshot 2023-10-28 160928.png}
  \caption{Image 8}
\end{figure}
\\
\begin{figure}[h]
  \centering
  \includegraphics[width=1.6\textwidth]{Screenshot 2023-10-28 160950.png}
  \caption{Image 9}
\end{figure}
\begin{figure}[h]
  \centering
  \includegraphics[width=1.6\textwidth]{Screenshot 2023-10-28 161015.png}
  \caption{Image 10}
\end{figure}
\\





\end{document}
